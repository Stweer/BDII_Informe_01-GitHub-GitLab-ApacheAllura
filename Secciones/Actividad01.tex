\section{Actividad No 01 – GitHub} 
GitHub es una importante herramienta para el acceso centralizado a los proyectos gracias a las ventajas que plantea el alojamiento de codigo. En este sentido, el servicio permite que varios participantes trabajen en un mismo proyecto a nivel global y que guarden sus cambios en cualquier momento de forma independiente. A diferencia de otros proveedores de servicios, en lo que respecta a la administracion de software open source, en GitHub el foco de atencion no se situa en el el proyecto como una compilacion de codigo fuente, sino en la posibilidad de hacer uso individualizado de los repositorios(Gestionados con Git). Los usuarios de Github pueden utilizar Git para gestionar, revisar y preparar sus proyectos de software

\begin{itemize}

	\item\textbf {Ventajas e Desventajas}

	Una ventaja importante de GitHub es que el servicio pone a disposici\'on de todos los usuarios repositorios de c\'odigo p\'ublicos y libres sin l\'imites. Sin embargo, el mantenimiento de repositorios privados está sujeto al pago de una suscripción mensual. GitHub también ofrece la posibilidad de crear “organizaciones” que hacen las veces de cuentas regulares a menos que tengas como mínimo una cuenta de usuario de tu propiedad.

A pesar de todo, en algunos casos puede haber ciertas limitaciones en lo relativo a la facilidad de uso y eficiencia de GitHub. En ocasiones surgen complicaciones entre el programa cliente y la compañ\'ia cuando, por ejemplo, un servidor privado opera como host para el código creado. Otra de las razones que motivan la elección de alternativas a GitHub es el empleo de un VCS diferente no soportado por GitHub. Hoy en d\'ia existen diversas alternativas a GitHub, pero en el presente artículo te hablamos de cinco de ellas.
	\begin{center}
	\includegraphics[width=17cm]{./Imagenes/git} 
	\end{center}


\end{itemize} 